
\chapter*{Összefoglalás}
\markboth{Összefoglalás}{Összefoglalás}
\addcontentsline{toc}{chapter}{Összefoglalás}

Osszefoglalas

A munka három fő témakörből áll. Tomor osszefoglalas.

\section*{1. resz}

1. resz rovid bevezeto.

1. resz ismertetes.

\section*{2. resz}

2. resz rovid bevezeto.

2. resz ismertetes.

\newpage
\section*{3. resz}

3. resz rovid bevezeto.

3. resz ismertetes.
\newpage
\section*{A disszertáció tézisei}

Az \textbf{első téziscsoportban} a hozzájárulásaim az 1. reszhez kapcsolódik. A részletes bemutatás a \ref{chapter_2}. fejezetben található.

\begin{enumerate}[wide = 0pt, widest = {III/5.}, leftmargin =*]
    \item[I/1.] Hozzajarulas I.
    
    \item[I/2.] Hozzajarulas II.

    \item[I/3.] Hozzajarulas III.
    
    \item[I/4.] Hozzajarulas IV.
\end{enumerate}

\noindent
A \textbf{második téziscsoport} 2. részhez kapcsolódik. A részletes bemutatás a \ref{chapter_3}. fejezetben található.

\begin{enumerate}[wide = 0pt, widest = {III/5.}, leftmargin =*]
    \item[II/1.] Hozzajarulas I.
    
    \item[II/2.] Hozzajarulas II.

    \item[II/3.] Hozzajarulas III.
    
    \item[II/4.] Hozzajarulas IV.
\end{enumerate}

\noindent
A \textbf{harmadik téziscsoport} hozzájárulásai a 3. részhez kapcsolódnak. Részletes bemutatás a \ref{chapter_4}. fejezetben található.

\begin{enumerate}[wide = 0pt, widest = {III/5.}, leftmargin =*]
       \item[III/1.] Hozzajarulas I.
    
    \item[III/2.] Hozzajarulas II.

    \item[III/3.] Hozzajarulas III.
    
    \item[III/4.] Hozzajarulas IV.
\end{enumerate}