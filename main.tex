\documentclass[12pt,a4paper,twoside]{book}
\usepackage[utf8]{inputenc}
%\usepackage[T1]{fontenc}
\usepackage{setspace}
\usepackage{amsmath,amsfonts,amssymb}
\usepackage{cite}
\usepackage{algorithm,algpseudocode}
\usepackage{setspace}
\usepackage{bibentry}
\usepackage[bottom]{footmisc}
\usepackage[
a4paper,
twoside,
bindingoffset=1cm,
inner=2.5cm,
%outer=2.5cm,
outer=2cm,
top=3.5cm,
%top=3.5cm
bottom=3.5cm,
%headsep=1.2cm
headsep=1cm
]{geometry}
%\usepackage{geometry}
%\geometry{a4paper,twoside,textwidth=6.0in,textheight=8.5in}
\usepackage{charter}
\usepackage[sort,numbers]{natbib}
%\usepackage{nyul_thesis}
\usepackage{booktabs,caption,multicol,hhline,tikz,multirow,array}
\usetikzlibrary{arrows.meta}
\usepackage{colortbl}
\usepackage{arydshln}
\usepackage{url}
\usepackage{subcaption}
\usepackage{caption}
\usepackage{graphicx, multirow, array,amsmath}
\usepackage{tabularx}
\usepackage{hyperref}
\usepackage{enumitem}
\usepackage[acronym, nopostdot]{glossaries}

\newcolumntype{C}[1]{>{\centering\arraybackslash}p{#1}}

% \newcolumntype{C}[1]{>{\centering\let\newline\\\arraybackslash\hspace{0pt}}m{#1}}
\newcolumntype{M}[1]{>{\centering\arraybackslash}m{#1}}
\newcolumntype{N}{@{}m{0pt}@{}}
\setlength{\belowcaptionskip}{10pt plus 3pt minus 2pt}
\DeclareMathOperator*{\argmin}{arg\,min}
\DeclareMathOperator*{\argmax}{arg\,max}

\usepackage{fancyhdr}
\fancypagestyle{plain}{%
	\fancyhead{}%
	\renewcommand{\headrulewidth}{0pt}%
	\renewcommand{\footrulewidth}{0pt}%
	\fancyfoot[C]{\bfseries\thepage}}
\fancypagestyle{mainmatter}{%
	\fancyhf{}
	\renewcommand{\headrulewidth}{0.4pt}%
	\renewcommand{\footrulewidth}{0pt}%
	\fancyhead[LE,RO]{\bfseries\thepage}
	\fancyhead[LO]{\bfseries\nouppercase\rightmark}
	\fancyhead[RE]{\bfseries\nouppercase\leftmark}}
\fancypagestyle{frontmatter}{%
	\fancyhead{}%
	\renewcommand{\headrulewidth}{0pt}%
	\renewcommand{\footrulewidth}{0pt}%
	\fancyfoot[C]{\bfseries\thepage}}

\captionsetup[figure]{labelfont={bf},textfont={it}}
\captionsetup[table]{labelfont={bf},textfont={it}}


%----------------------------------------------------------------------------------------
%	THESIS INFORMATION
%----------------------------------------------------------------------------------------

\makeglossaries

\begin{document}
%\UseRawInputEncoding

\setstretch{1.1}

\nobibliography*
\newcounter{cont}

% Title

\thispagestyle{empty}

\begin{center}
%	\vfill
	\vspace*{0.25cm}
	
	\begin{spacing}{2}
		{\Huge \textbf{Title of Thesis}}
	\end{spacing}
	
	\vspace*{2cm}
	
	{\Large PhD Thesis}
	
%	\vspace{2.5cm}
	\vfill
	
	{\Large Firstname Lastname}
	
	\vspace{0.25cm}
	
	{\Large Supervisor: Firstname Lastname, PhD}
	
%	\vspace{2cm}
	\vfill
	
	{\large Doctoral School of DSCHOOL}
	
	\vspace{0.25cm}
	
	{\large Department of DEPART}
	
	\vspace{0.25cm}
	
	{\large Faculty of Science and Informatics}
	
	\vspace{0.25cm}
	
	{\large University of Szeged}
	
%	\vspace{4cm}
	\vfill
	
	\includegraphics[width=0.3\linewidth]{Figures/szte_logo}
	
	\vfill
	
	{\large Szeged
		
		 DATE}
	
\end{center}
 


\newpage
\thispagestyle{empty}
\mbox{}

% \newpage
% \thispagestyle{empty}
% \begin{flushright}
% \parbox[t]{10cm}{\textit{"Scientists study the world as it is,}\\ \textit{\hphantom{"}engineers  create the world that never has been."}} \\
% \vspace{1em}
% \parbox[t]{5cm}{\small(Kármán Tódor)}
% \end{flushright}

% \newpage
% \thispagestyle{empty}
% \mbox{}

\frontmatter
\renewcommand{\chaptermark}[1]{\markboth{#1}{}}
\renewcommand{\sectionmark}[1]{\markright{\thesection\ #1}}
\pagestyle{frontmatter}

\tableofcontents

% Empty pages

\newpage
\thispagestyle{empty}
\mbox{}

% \newpage
% \thispagestyle{empty}
% \mbox{}

% Main

\mainmatter
\renewcommand{\chaptermark}[1]{\markboth{#1}{}}
\renewcommand{\sectionmark}[1]{\markright{\thesection\ #1}}
\pagestyle{mainmatter}

% List of Figures
\listoffigures
% List of Tables
\listoftables

% List of Acronyms
\newacronym{gcd}{GCD}{Greatest Common Divisor}
\newacronym{lcm}{LCM}{Least Common Multiple}

% Acronym List
\printglossary[type=\acronymtype, title=Abbreviations]

% Intro:
\chapter{Introduction}

Introduction
\vfill
\newpage

Intro cont.
\section{Contributions}

The ideas, figures, tables and results included in this thesis were published in scientific papers (listed at the end of the thesis). In a nutshell, the author is responsible for the following contributions:

\vspace*{1em}

\noindent
\textbf{Chapter 2.}: Brief Contrib.

\vspace*{1em}

\noindent
\textbf{Chapter 3.}: Brief Contrib.

\vspace*{1em}

\noindent
\textbf{Chapter 4.}: Brief Contrib.





\section{Examples}

% Acronym examples
Given a set of numbers, there are elementary methods to compute 
its \acrlong{gcd}, which is abbreviated \acrshort{gcd}. This process 
is similar to that used for the \acrfull{lcm}.

% Figure example:
Examples of figure:
\begin{figure}[h]
\caption{Example of figure}
\centering
\includegraphics[width=0.3\linewidth]{Figures/szte_logo}
\end{figure}

% Table example:
Examples of table:
\begin{table}[h]
\centering
\begin{tabular}{|l|l|l|}
\hline
\#    & Col 1 & Col2 \\ \hline
Row 1 & 0     & 1    \\ \hline
Row 2 & 2     & 3    \\ \hline
\end{tabular}
\caption{Example of table}
\label{main:table_1}
\end{table}

Example of equations:
% Equations example
\begin{equation}
    x_{test} = b^2 + c^2
\end{equation}

% Thesis:
\chapter{Name of Chapter II}
\label{chapter_2}

Chapter brief.

\section{Introduction}
Chapter intro.
\textit{Structure of the chapter}: Section REF summarizes the related works and briefly presents previous results. The Work I is covered in Section REF. The Work II is covered in Section REF. Result and final thoughts are summarized in Section~\ref{ch2:contrib}.
\section{Related Works}
Chapter related research.

\section{Chapter Work I.}
Work I. \cite{b:isis-tracking-techreport}
\section{Chapter Work II.}
Work II. \cite{b:mays01}
\section{Discussion and concluding remarks}
\label{ch2:contrib}

Chapter conclusion.

The author of this PhD thesis is responsible for the following contributions presented in this chapter:
\begin{enumerate}[wide = 0pt, widest = {II/5.}, leftmargin =*]
	\item[II/1.] Contribution I.
	
    \item[II/2.] Contribution II.
	
    \item[II/3.] Contribution III.
	
    \item[II/4.] Contribution X
\end{enumerate} 
\chapter{Name of Chapter III}
\label{chapter_3}

Chapter brief.

\section{Introduction}
Chapter intro.
\textit{Structure of the chapter}: Section REF summarizes the related works and briefly presents previous results. The Work I is covered in Section REF. The Work II is covered in Section REF. Result and final thoughts are summarized in Section~\ref{ch3:contrib}.
\section{Related Works}
Chapter related research.

\section{Chapter Work I.}
Work I. \cite{b:isis-floodrouting}
\section{Chapter Work II.}
Work II.
\section{Discussion and concluding remarks}
\label{ch3:contrib}

Chapter conclusion.

The author of this PhD thesis is responsible for the following contributions presented in this chapter:
\begin{enumerate}[wide = 0pt, widest = {II/5.}, leftmargin =*]
	\item[II/1.] Contribution I.
	
    \item[II/2.] Contribution II.
	
    \item[II/3.] Contribution III.
	
    \item[II/4.] Contribution X
\end{enumerate}
\include{Chapters/Thesis3/chapter_4}
% References:
\include{Chapters/references}

\backmatter

% Bibliography:
\bibliographystyle{plain}
\addcontentsline{toc}{chapter}{Bibliography}
\bibliography{refs}

% Summary
\chapter*{Summary}
\markboth{Summary}{Summary}
\addcontentsline{toc}{chapter}{Summary}

The PhD thesis presents ... 

The dissertation consists of three major parts. In Chapter...

\section*{Work I.}

Intro

In Chapter~\ref{chapter_2}, brief...

\section*{Work II.}

Intro

In Chapter~\ref{chapter_3}, brief...

\newpage
\section*{Work III.}

Intro

In Chapter~\ref{chapter_4}, brief...

\vfill
\pagebreak

\section*{Contributions of the thesis}

In the \textbf{first thesis group}, the contributions are related to .. . Detailed discussion can be found in Chapter~\ref{chapter_2}.

\begin{enumerate}[wide = 0pt, widest = {I/4.}, leftmargin =*]
    \item[I/1.] Contrib I.
    
    \item[I/2.] Contrib II.
    
    \item[I/3.] Contrib III.
    
    \item[I/4.] Contrib IV.
\end{enumerate}

\vspace{1cm}

\noindent
In the \textbf{second thesis group}, the contributions are related to the .. . Detailed discussion can be found in Chapter~\ref{chapter_3}.

\begin{enumerate}[wide = 0pt, widest = {II/5.}, leftmargin =*]
    \item[II/1.] Contrib I.
    
    \item[II/2.] Contrib II.
    
    \item[II/3.] Contrib III.
    
    \item[II/4.] Contrib IV.
\end{enumerate}

\vfill
\pagebreak

\noindent
In the \textbf{third thesis group}, the contributions are related to the .. . Detailed discussion can be found in Chapter~\ref{chapter_4}.

\begin{enumerate}[wide = 0pt, widest = {III/5.}, leftmargin =*]
    \item[III/1.] Contrib I.
    
    \item[III/2.] Contrib II.
    
    \item[III/3.] Contrib III.
    
    \item[III/4.] Contrib IV.
\end{enumerate}

\chapter*{Összefoglalás}
\markboth{Összefoglalás}{Összefoglalás}
\addcontentsline{toc}{chapter}{Összefoglalás}

Osszefoglalas

A munka három fő témakörből áll. Tomor osszefoglalas.

\section*{1. resz}

1. resz rovid bevezeto.

1. resz ismertetes.

\section*{2. resz}

2. resz rovid bevezeto.

2. resz ismertetes.

\newpage
\section*{3. resz}

3. resz rovid bevezeto.

3. resz ismertetes.
\newpage
\section*{A disszertáció tézisei}

Az \textbf{első téziscsoportban} a hozzájárulásaim az 1. reszhez kapcsolódik. A részletes bemutatás a \ref{chapter_2}. fejezetben található.

\begin{enumerate}[wide = 0pt, widest = {III/5.}, leftmargin =*]
    \item[I/1.] Hozzajarulas I.
    
    \item[I/2.] Hozzajarulas II.

    \item[I/3.] Hozzajarulas III.
    
    \item[I/4.] Hozzajarulas IV.
\end{enumerate}

\noindent
A \textbf{második téziscsoport} 2. részhez kapcsolódik. A részletes bemutatás a \ref{chapter_3}. fejezetben található.

\begin{enumerate}[wide = 0pt, widest = {III/5.}, leftmargin =*]
    \item[II/1.] Hozzajarulas I.
    
    \item[II/2.] Hozzajarulas II.

    \item[II/3.] Hozzajarulas III.
    
    \item[II/4.] Hozzajarulas IV.
\end{enumerate}

\noindent
A \textbf{harmadik téziscsoport} hozzájárulásai a 3. részhez kapcsolódnak. Részletes bemutatás a \ref{chapter_4}. fejezetben található.

\begin{enumerate}[wide = 0pt, widest = {III/5.}, leftmargin =*]
       \item[III/1.] Hozzajarulas I.
    
    \item[III/2.] Hozzajarulas II.

    \item[III/3.] Hozzajarulas III.
    
    \item[III/4.] Hozzajarulas IV.
\end{enumerate}
\chapter*{Publications}
\markboth{Publications}{Publications}
\addcontentsline{toc}{chapter}{Publications}

\vspace*{2em}

\subsection*{Journal publications}

\vspace*{1em}

\begin{enumerate}[wide = 0pt, widest = {[4]}, leftmargin =*]

\item[{[1]}] \textbf{G. Z}, A. A, G. G, G. G, and L.~G. N.
\newblock {Paper Title}.
\newblock \emph{Journal Name}, VOL(NUMBER), PP-PP, YEAR.

\end{enumerate}

\subsection*{Full papers in conference proceedings}
\vspace*{1em}

\begin{enumerate}[wide = 0pt, widest = {[4]}, leftmargin =*]
\item[{[5]}] \textbf{G. Z}, G. W, P. V, H.B. R, M. M, Á. L.
\newblock {Paper Title}.
\newblock In \emph{Proceedings of the Xth Annual International Conference on Conference}, DISTRIBUTOR, PP-PP, YEAR.
\end{enumerate}

\subsection*{Further related publications}
\vspace*{1em}

\begin{enumerate}[wide = 0pt, widest = {[15]}, leftmargin =*]

\item[{[8]}] \textbf{G. Z}, A. B, G. K, G. H, and L.~G. N.
\newblock {Paper Title}.
\newblock In \emph{The Xth Jubilee Conference of Something: Volume of extended abstracts.}, YEAR.

\end{enumerate}

% Acknowledgement
\chapter*{Acknowledgments}
\markboth{}{}

First of all, I would like to thank my supervisor, Firstname Lastname, for directing my PhD studies. I would also like to thank my colleagues and friends who helped me to realize the results presented here and to enjoy the period of my studies. 
Last, but not least, I wish to thank my wife and family for their constant love and support. 

% Empty page
\thispagestyle{empty}

\end{document}  
